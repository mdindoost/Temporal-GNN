\section{Conclusion}
\label{sec:conclusion}

}
# 6. Conclusion

We presented TempAnom-GNN, the first temporal graph neural network designed specifically for real-time anomaly detection in evolving networks. Through comprehensive evaluation on Bitcoin trust networks, we established a fundamental distinction between retrospective analysis and prospective deployment that has important implications for both methodology and evaluation practices.

\subsection{6.1 Key Contributions and Findings

\textbf{Methodological Innovation\textbf{: Our architecture integrates temporal evolution encoding, memory mechanisms, and trajectory prediction to address real-world deployment requirements. The unified framework provides a foundation for future temporal anomaly detection research.

\textbf{Deployment Validation\textbf{: Statistical validation across deployment scenarios demonstrates significant advantages: 20.8% improvement in early detection (p < 0.0001) and 13.2% improvement for cold start users (p = 0.0017). These improvements address critical gaps in existing methods.

\textbf{Architectural Insights\textbf{: Component analysis reveals that evolution-only architectures outperform complex combinations, providing actionable guidance for practitioners. This finding challenges common assumptions about component synergy in temporal graph methods.

\textbf{Evaluation Framework\textbf{: Our distinction between retrospective analysis and prospective deployment advances evaluation practices for temporal graph research, highlighting the importance of deployment-focused metrics.

\subsection{6.2 Practical Impact

\textbf{For Practitioners\textbf{: We provide clear guidance on when to use simple statistical methods (retrospective analysis) versus temporal graph approaches (prospective deployment). The component analysis offers architectural recommendations for production systems.

\textbf{For Researchers\textbf{: Our evaluation framework and statistical validation methodology provide a template for rigorous temporal graph evaluation that goes beyond retrospective metrics.

\textbf{For Domain Experts\textbf{: The Bitcoin network analysis demonstrates practical fraud detection capabilities with statistically validated improvements in real-world scenarios.

\subsection{6.3 Deployment Guidance Summary

\textbf{Method Selection\textbf{:
- Use simple baselines for retrospective analysis with complete data
- Use TempAnom-GNN for real-time deployment scenarios
- Consider evolution-only components for optimal deployment performance

\textbf{Implementation Recommendations\textbf{:
- Start with single-component architectures
- Validate component combinations on deployment scenarios
- Prioritize early detection and cold start capabilities for production systems

\subsection{6.4 Future Research Directions

\textbf{Cross-Domain Validation\textbf{: Extend evaluation to social networks, e-commerce platforms, and other financial systems to establish generalizability.

\textbf{Advanced Integration Methods\textbf{: Research approaches to reduce component interference while maintaining individual component strengths.

\textbf{Real-Time Optimization\textbf{: Develop streaming algorithms optimized for production deployment with minimal latency requirements.

\textbf{Interpretability Enhancement\textbf{: Create tools to explain temporal pattern decisions for regulatory compliance and user trust.

\subsection{6.5 Broader Implications

Our work demonstrates that effective temporal graph research requires careful consideration of deployment requirements, not just algorithmic innovation. The distinction between retrospective analysis and prospective deployment provides a framework for more realistic evaluation and more practical method development.

\textbf{For the Temporal Graph Community\textbf{: We encourage adoption of deployment-focused evaluation that includes early detection, cold start scenarios, and statistical validation with confidence intervals.

\textbf{For the Anomaly Detection Community\textbf{: Our findings highlight the value of temporal patterns for deployment scenarios while acknowledging the continued importance of simple baselines for retrospective analysis.

\textbf{For the Broader ML Community\textbf{: The component interference phenomenon and deployment-focused evaluation methodology have implications beyond temporal graph research, contributing to more principled and practical machine learning research practices.

TempAnom-GNN advances the state-of-the-art in temporal graph anomaly detection while providing practical guidance for real-world deployment. Our comprehensive evaluation framework and architectural insights pave the way for more effective and deployable temporal graph methods.

\subsection{Acknowledgments

We thank the reviewers for their constructive feedback and the Stanford SNAP team for providing the Bitcoin trust network datasets that enabled this research.

\subsection{Data and Code Availability

The Bitcoin Alpha and OTC datasets are publicly available from Stanford SNAP. Our implementation and experimental code will be released upon acceptance to support reproducibility and future research.
